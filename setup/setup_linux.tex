\documentclass{article}
\usepackage[margin=1.0in]{geometry}
\usepackage[utf8]{inputenc}

\title{\LaTeX \ Setup for the Sublime Editor with Linux}
\author{Joseph Barton}
\date{}

\begin{document}

\maketitle

\begin{enumerate}
	\item Install Tex Live, other fonts, and other useful packages. (texlive-science has algorithm related packages such as algorithm.sty and algorithmic.sty)

	\$ sudo apt install texlive-base

	\$ sudo apt install texlive-fonts-recommended

	\$ sudo apt install texlive-fonts-extra

	\$ sudo apt install texlive-science

	\$ sudo apt install latexmk

	\$ sudo apt install biber

\item In Sublime, go to Tools $\rightarrow$ Install Package Control to install the package controller.

\item Use Sublime's package control to install LaTeXTools package. In Sublime, go to ``Preferences $\rightarrow$ Package Control $\rightarrow$ Package Control: Install Package'' and scroll or search for LaTeXTools

\item Ensure setup is correct:
	\begin{itemize}
	\item Go to ``Preferences $\rightarrow$ Package Settings $\rightarrow$ LaTeXTools $\rightarrow$ Settings -- User'' and scroll to the linux settings (about line 227)
	\item You will likely need to change the texpath to ``\$PATH:/usr/bin'' because that's where stuff is installed.
	\item Go to ``Preferences $\rightarrow$ Package Settings $\rightarrow$ LaTeXTools $\rightarrow$ Check System''. The status of all essential programs should be labeled as ``available''. The packages that have a ``missing'' status may need to be installed.  You may be able to get away without installing xelatex and magick.
	\end{itemize}

\item Now Create a .tex document (or just use this document) and press Ctrl+b to compile the file. The default pdf viewer should pop up and show the compiled document.

\item Install LaTex-cwl and LaTeXYZ with Sublime's package control just like we did above. These packages assist with autocompletion and syntax shortcuts. For reference:
	\begin{itemize}
	\item https://github.com/LaTeXing/LaTeX-cwl
	\item https://github.com/randy3k/LaTeXYZ
	\end{itemize}

\item A comprehensive list of \LaTeX \ symbols can be found here:
	\begin{itemize}
	\item https://mirror.its.dal.ca/ctan/info/symbols/comprehensive/symbols-a4.pdf
	\end{itemize}
\end{enumerate}

\end{document}