\documentclass{article}
\usepackage[margin=1.0in]{geometry}
\usepackage[utf8]{inputenc}

\title{\LaTeX \ Setup for the Sublime Editor with Windows}
\author{Joseph Barton}
\date{}

\begin{document}

\maketitle

\begin{enumerate}
\item Install TeX Live https://www.tug.org/texlive/acquire.html. This will take some time to download.
	\begin{enumerate}
		\item TeX Live is not necessary on personal computers. MikTeX is restricted on ARA systems, but is much easier to deal with on personal computers.
		\item Install MikTeX. Install to your user profile (should be the default), and it will automatically be added to your user path.
	\end{enumerate}

\item Add TeX Live executables to your system path. It is in a folder similar to ``C:$ \backslash $texlive$ \backslash $2020$ \backslash $bin$ \backslash $win32''
	\begin{enumerate}
		\item Unnecessary for MikTeX install
	\end{enumerate} 


\item In Sublime, go to Tools $\rightarrow$ Install Package Control to install the package controller.

\item Use Sublime's package control to install LaTeXTools package. In Sublime, go to ``Preferences $\rightarrow$ Package Control $\rightarrow$ Package Control: Install Package'' and scroll or search for LaTeXTools

\item Install the lightweight pdf reader SumatraPDF at https://www.sumatrapdfreader.org/download-free-pdf-viewer.html
	\begin{enumerate}
		\item If on an ARA system, it will likely install in your Admin folder and give you an error that it can't be opened.  It may be in ``C:$ \backslash $Users$ \backslash $.ARA-ADMIN$ \backslash $AppData$ \backslash $Local''. So move the SumatraPDF folder to ``C:$ \backslash $Program Files'' or somewhere else that is accessible.
	\end{enumerate}

\item Add the ``C:$ \backslash $Program Files$ \backslash $SumatraPDF'' (or whatever you chose) to your User path and restart your computer for the path changes to take affect.

\item Ensure setup is correct:
	\begin{enumerate}
		\item Go to ``Preferences $\rightarrow$ Package Settings $\rightarrow$ LaTeXTools $\rightarrow$ Settings -- User'' and scroll to the Platform Settings (about line 209)
		\item Ensure the TeX distro is set to ``texlive''
		\item Go to ``Preferences $\rightarrow$ Package Settings $\rightarrow$ LaTeXTools $\rightarrow$ Check System''. The status of all essential programs should be labeled as ``available''.
	\end{enumerate}

\item Now Create a .tex document (or just use this document) and press Ctrl+b to compile the file. Sumatra should pop up and show you the compiled document.

\item Install LaTex-cwl and LaTeXYZ with Sublime's package control just like we did above. These packages assist with autocompletion and syntax shortcuts. For reference:
	\begin{itemize}
		\item https://github.com/LaTeXing/LaTeX-cwl
		\item https://github.com/randy3k/LaTeXYZ
	\end{itemize}

\item Configure inverse search so you can compare a pdf object to an editor object easily. (First try to see if inverse search already works before doing this step.)
	\begin{enumerate}
		\item Compile a tex file so that SumatraPDF is displaying a tex document.
		\item In the SumatraPDF window, go to ``Settings $\rightarrow$ Options'' and in the inverse search section put the path to your sublime executible followed by ``\%f:\%l''
		\item For example: ``C:$ \backslash $Program Files$ \backslash $Sublime Text 3$ \backslash $sublime\_text.exe'' ``\%f:\%l''
		\item Now double clicking on a line in the pdf will automatically move the cursor to the corresponding line in Sublime.
	\end{enumerate}

\item A comprehensive list of \LaTeX \ symbols can be found here:
	\begin{itemize}
		\item https://mirror.its.dal.ca/ctan/info/symbols/comprehensive/symbols-a4.pdf
	\end{itemize}
\end{enumerate}

\end{document}